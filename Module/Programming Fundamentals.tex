%%%%%%%%%%%%%%%%%%%%%%%%%%%%%%%%%%%%%%%%%%%%%%%%%%%
%% LaTeX book template                           %%
%% Author:  Amber Jain (http://amberj.devio.us/) %%
%% License: ISC license                          %%
%%%%%%%%%%%%%%%%%%%%%%%%%%%%%%%%%%%%%%%%%%%%%%%%%%%

\documentclass[a4paper,11pt,oneside]{book}
\usepackage{modulestyle}

%%%%%%%%%%%%%%%%%%%%%%%%%%%%%%%%%%%%%%%%%%%%%%%%%%%%%%%%%
% Source: http://en.wikibooks.org/wiki/LaTeX/Hyperlinks %
%%%%%%%%%%%%%%%%%%%%%%%%%%%%%%%%%%%%%%%%%%%%%%%%%%%%%%%%%

%%%%%%%%%%%%%%%%%%%%%%%%%%%%%%%%%%%%%%%%%%%%%%%%%%%%%%%%%%%%%%%%%%%%%%%%%%%%%%%%
% 'dedication' environment: To add a dedication paragraph at the start of book %
% Source: http://www.tug.org/pipermail/texhax/2010-June/015184.html            %
%%%%%%%%%%%%%%%%%%%%%%%%%%%%%%%%%%%%%%%%%%%%%%%%%%%%%%%%%%%%%%%%%%%%%%%%%%%%%%%%
\newenvironment{dedication}
{
   \cleardoublepage
   \thispagestyle{empty}
   \vspace*{\stretch{1}}
   \hfill\begin{minipage}[t]{0.66\textwidth}
   \raggedright
}
{
   \end{minipage}
   \vspace*{\stretch{3}}
   \clearpage
}

%%%%%%%%%%%%%%%%%%%%%%%%%%%%%%%%%%%%%%%%%%%%%%%%
% Chapter quote at the start of chapter        %
% Source: http://tex.stackexchange.com/a/53380 %
%%%%%%%%%%%%%%%%%%%%%%%%%%%%%%%%%%%%%%%%%%%%%%%%
\makeatletter
\renewcommand{\@chapapp}{}% Not necessary...
\newenvironment{chapquote}[2][2em]
  {\setlength{\@tempdima}{#1}%
   \def\chapquote@author{#2}%
   \parshape 1 \@tempdima \dimexpr\textwidth-2\@tempdima\relax%
   \itshape}
  {\par\normalfont\hfill--\ \chapquote@author\hspace*{\@tempdima}\par\bigskip}
\makeatother

%%%%%%%%%%%%%%%%%%%%%%%%%%%%%%%%%%%%%%%%%%%%%%%%%%%
% First page of book which contains 'stuff' like: %
%  - Book title, subtitle                         %
%  - Book author name                             %
%%%%%%%%%%%%%%%%%%%%%%%%%%%%%%%%%%%%%%%%%%%%%%%%%%%

\newcommand{\BookTitle}{Programming Fundamentals}
\newcommand{\BookTitleFootnote}{A course in the Bachelor of Technical Vocational Teacher Education (BTVTED) Major in Computer System Servicing.}

\newcommand{\BookSubtitle}{A Study Guide for Students of Sorsogon State University - Bulan Campus}
\newcommand{\BookSubtitleFootnote}{This book is a study guide for students of
Sorsogon State University - Bulan Campus taking up the course Programming Fundamentals.}

\newcommand{\BookAuthorFirstName}{Jarrian Vince}
\newcommand{\BookAuthorLastName}{Gojar}
\newcommand{\BookAuthorName}{Jarrian Vince G. Gojar}
\newcommand{\BookAuthorURL}{https://github.com/godkingjay}

% Book's title and subtitle
\title{\Huge \textbf{\BookTitle}  \footnote{\BookTitleFootnote} \\
\huge \BookSubtitle \footnote{\BookSubtitleFootnote}}

% Author
\author{\textsc{\BookAuthorName}\thanks{\url{\BookAuthorURL}}}

\begin{document}

\frontmatter
\maketitle

%%%%%%%%%%%%%%%%%%%%%%%%%%%%%%%%%%%%%%%%%%%%%%%%%%%%%%%%%%%%%%%
% Add a dedication paragraph to dedicate your book to someone %
%%%%%%%%%%%%%%%%%%%%%%%%%%%%%%%%%%%%%%%%%%%%%%%%%%%%%%%%%%%%%%%
\begin{dedication}
Sorsogon State University - Bulan Campus
\end{dedication}

%%%%%%%%%%%%%%%%%%%%%%%%%%%%%%%%%%%%%%%%%%%%%%%%%%%%%%%%%%%%%%%%%%%%%%%%
% Auto-generated table of contents, list of figures and list of tables %
%%%%%%%%%%%%%%%%%%%%%%%%%%%%%%%%%%%%%%%%%%%%%%%%%%%%%%%%%%%%%%%%%%%%%%%%
\tableofcontents
\listoffigures
\listoftables
\lstlistoflistings

\mainmatter

%%%%%%%%%%%
% Preface %
%%%%%%%%%%%
\chapter*{Preface}
% A Quote all about Programming Fundamentals
\begin{chapquote}{Donald Ervin Knuth}
``Programming is the art of telling another human being what one wants
the computer to do.''
\end{chapquote}

\noindent \BookAuthorName \\
\noindent \url{\BookAuthorURL}

%%%%%%%%%%%%%%%%%%%%%%%%%%%%%%%%%%%%
%%%%%~ NEW CHAPTER STARTS HERE %%%%%
%%%%%%%%%%%%%%%%%%%%%%%%%%%%%%%%%%%%
\chapter{Introduction to Python}

\section{Introduction}

\textbf{Python} is a high-level, interpreted, interactive, and
object-oriented scripting language. It is designed to be highly
readable, using English keywords frequently, unlike other languages
that use punctuation, and it has fewer syntactical constructions.

Python is a versatile, general-purpose, and powerful programming language.
It is an excellent first language because it is concise and easy to read.
Whether you want to do web development, machine learning, or data science,
Python is the language for you.

\subsection{History of Python}

Python was developed by \textbf{Guido van Rossum} in the late 80's
and early 90's at the National Research Institute for Mathematics and
Computer Science in the Netherlands. Python is derived from many other
languages, including ABC, Modula-3, C, C++, Algol-68, SmallTalk, and
Unix shell and other scripting languages.

\subsection{What is Programming Language?}

A programming language is a formal language comprising a set of instructions
that produce various kinds of output. Programming languages are used in
computer programming to implement algorithms.

\section{Environment Setup}

In software development, an \textbf{integrated development environment (IDE)}
is a software application that provides comprehensive facilities to computer
programmers for software development. An IDE normally consists of a source
code editor, build automation tools, and a debugger. Other than the IDE,
it is also important to have the programming language installed on your
computer.

\subsection{How to Download and Install Python on Windows}

To download and install Python on Windows, follow these steps:

\begin{enumerate}
    \item Open a web browser and go to the official Python website at
  \url{https://www.python.org/downloads/}.
    \item Click on the latest version of Python to download the installer.
    \item Run the installer and follow the installation wizard.
    \item Make sure to check the box that says ``Add Python to PATH''.
    \item Click on the ``Install Now'' button to install Python on your computer.

    \begin{figure}[!h]
        \centering
        \includegraphics[width=0.8\textwidth]{./assets/setup/python-install.png}
        \caption{Python Installation Wizard}
        \label{fig:python-install}
    \end{figure}
    
    \item Once the installation is complete, you can verify the installation
    by opening a command prompt and typing \texttt{python --version}.
    
    \begin{figure}[!h]
        \centering
        \includegraphics[width=0.8\textwidth]{./assets/setup/python-check.png}
        \caption{Python Installation Verification}
        \label{fig:python-check}
    \end{figure}
\end{enumerate}

\subsection{How to Download PyCharm IDE on Windows}

To download and install PyCharm IDE on Windows, follow these steps:

\begin{enumerate}
    \item Open a web browser and go to the official PyCharm website at
    \url{https://www.jetbrains.com/pycharm/download/}.
    \item Click on the ``Download'' button to download the installer.

    \begin{figure}[!h]
        \centering
        \includegraphics[width=0.8\textwidth]{./assets/setup/pycharm-community.png}
        \caption{PyCharm Community Edition Download}
        \label{fig:pycharm-community}
    \end{figure}
    
    \item Run the installer and follow the installation wizard.
    \item Make sure to check the box that says ``Create associations''.
    \item Click on the ``Install'' button to install PyCharm on your computer.
    
    \begin{figure}[!h]
        \centering
        \includegraphics[width=0.8\textwidth]{./assets/setup/pycharm-install.png}
        \caption{PyCharm Community Edition Installation Wizard}
        \label{fig:pycharm-install}
    \end{figure}

    \item Once the installation is complete, you can launch PyCharm by
    clicking on the desktop shortcut or searching for it in the Start menu.
\end{enumerate}

\section{First Python Program}

In Python, the \texttt{print()} function is used to display output on the
screen. To write your first Python program, follow these steps:

\begin{enumerate}
    \item Open PyCharm IDE on your computer.
    \item Click on the ``Create New Project'' button to create a new
    project.
    \item Enter a name for your project and click on the ``Create'' button.
    \item Right-click on the project folder in the Project view and select
    ``New'' $\rightarrow$ ``Python File''.
    \item Enter a name for your Python file and click on the ``OK'' button.
    \item Type the following code in the Python file:
    
    \begin{lstlisting}[language=Python, caption={Hello, World! Program}, label={lst:hello-world}]
    print("Hello, World!")
    \end{lstlisting}
    
    \item Click on the ``Run'' button to run the program.
    \item You should see the output ``Hello, World!'' displayed in the Run window.
\end{enumerate}

\section{Basic Syntax}

In Python, the syntax refers to the rules that define the combinations of
symbols that are considered to be correctly structured programs in the
language. The Python syntax is simple and easy to learn. It is based on
indentation and does not require the use of curly braces or semicolons.
It can also be easily understood by beginners as it is very similarly
structured to the English language.

\subsection{Variables}

A variable is a name that refers to a value. In Python, variables are
created when you assign a value to them. You can assign a value to a
variable using the assignment operator \texttt{=}.

\begin{lstlisting}[language=Python, caption={Variable Assignment}, label={lst:variables}]
x = 5
y = "Hello, World!"

print(x)
print(y)
\end{lstlisting}

Code \ref{lst:variables} shows how to assign values to variables in Python.
In this example, the variable \texttt{x} is assigned the value 5, and the
variable \texttt{y} is assigned the value ``Hello, World!''.

\subsection{Python Identifiers}

An identifier is a name given to entities like class, functions, variables, etc.
It helps to differentiate one entity from another. In Python, an identifier
is a name used to identify a variable, function, class, module, or other
object. An identifier must start with a letter or an underscore (\_),
followed by letters, digits, or underscores.

\begin{lstlisting}[language=Python, caption={Valid and Invalid Identifiers}, label={lst:identifiers}]
# Valid Identifiers
my_variable = 5
myVariable = 10
_my_variable = 15

# Invalid Identifiers
1variable = 20
my-variable = 25
my variable = 30
\end{lstlisting}

Code \ref{lst:identifiers} shows examples of valid and invalid identifiers
in Python. In this example, the variables \texttt{my\_variable},
\texttt{myVariable}, and \texttt{\_my\_variable} are valid identifiers,
while the variables \texttt{1variable}, \texttt{my-variable}, and
\texttt{my variable} are invalid identifiers.

\subsection{Reserved Words}

Python has a set of reserved words that cannot be used as identifiers.
These reserved words are used by the Python interpreter to recognize the
structure of the program. Some of the reserved words in Python include:

\begin{multicols}{4}
    \begin{itemize}
        \item \texttt{False}
        \item \texttt{None}
        \item \texttt{True}
        \item \texttt{and}
        \item \texttt{as}
        \item \texttt{assert}
        \item \texttt{break}
        \item \texttt{class}
        \item \texttt{continue}
        \item \texttt{def}
        \item \texttt{del}
        \item \texttt{elif}
        \item \texttt{else}
        \item \texttt{except}
        \item \texttt{exec}
        \item \texttt{finally}
        \item \texttt{for}
        \item \texttt{from}
        \item \texttt{global}
        \item \texttt{if}
        \item \texttt{import}
        \item \texttt{in}
        \item \texttt{is}
        \item \texttt{lambda}
        \item \texttt{nonlocal}
        \item \texttt{not}
        \item \texttt{or}
        \item \texttt{pass}
        \item \texttt{print}
        \item \texttt{raise}
        \item \texttt{return}
        \item \texttt{try}
        \item \texttt{while}
        \item \texttt{with}
        \item \texttt{yield}
    \end{itemize}
\end{multicols}

These reserved words cannot be used as identifiers in Python. They cannot
be used as variable names, function names, or any other identifier names.

\subsection{Quotation in Python}

In Python, you can use either single quotes (\texttt{'}), double quotes
(\texttt{"}), or triple quotes (\texttt{'''} or \texttt{"""}). Single and
double quotes are used to represent strings, while triple quotes are used
to represent multi-line strings.

\begin{lstlisting}[language=Python, caption={Quotation in Python}, label={lst:quotation}]
single_quote = 'Hello, World!'
double_quote = "Hello, World!"
triple_quote = '''Hello,
World!'''

print(single_quote)
print(double_quote)
print(triple_quote)
\end{lstlisting}

Code \ref{lst:quotation} shows examples of using single quotes, double
quotes, and triple quotes in Python. In this example, the variables
\texttt{single\_quote}, \texttt{double\_quote}, and \texttt{triple\_quote}
are assigned string values using single quotes, double quotes, and triple
quotes, respectively. A single quote or double quote can be used to
represent a string value, while triple quotes are used to represent
multi-line strings.

\subsubsection{Escape Characters}

An escape character is a backslash (\textbackslash) followed by a character
that has a special meaning in Python. It is used to represent characters
that are difficult or impossible to type directly. Some of the common
escape characters in Python include:

\begin{multicols}{4}
    \begin{itemize}
        \item \texttt{\textbackslash n} - New Line
        \item \texttt{\textbackslash t} - Tab
        \item \texttt{\textbackslash r} - Carriage Return
        \item \texttt{\textbackslash \textbackslash} - Backslash
        \item \texttt{\textbackslash '} - Single Quote
        \item \texttt{\textbackslash "} - Double Quote
        \item \texttt{\textbackslash b} - Backspace
        \item \texttt{\textbackslash f} - Form Feed
    \end{itemize}
\end{multicols}

These escape characters are used to represent special characters in Python.
For example, the escape character \texttt{\textbackslash n} is used to
represent a newline character, while the escape character \texttt{\textbackslash t}
is used to represent a tab character.

\begin{lstlisting}[language=Python, caption={Escape Characters in Python}, label={lst:escape}]
new_line = "Hello,\nWorld!"
tab_space = "Hello,\tWorld!"
back_slash = "Hello,\\World!"
single_quote = "Hello,\'World!"
double_quote = "Hello,\"World!"
back_space = "Hello,\bWorld!"
form_feed = "Hello,\fWorld!"

print(new_line)
print(tab_space)
print(back_slash)
print(single_quote)
print(double_quote)
print(back_space)
print(form_feed)
\end{lstlisting}

Code \ref{lst:escape} shows examples of using escape characters in Python.
In this example, the variables \texttt{new\_line}, \texttt{tab\_space},
\texttt{back\_slash}, \texttt{single\_quote}, \texttt{double\_quote},
\texttt{back\_space}, and \texttt{form\_feed} are assigned string values
using escape characters to represent special characters.

\subsubsection{Multi-line Strings in Single or Double Quotes}

In Python, you can use triple quotes (\texttt{'''} or \texttt{"""}) to
represent multi-line strings. This allows you to write strings that span
multiple lines without using escape characters. However, if you want to
represent multi-line strings using single or double quotes, you can use
the escape character \texttt{\textbackslash n} to represent a newline
character.

\begin{lstlisting}[language=Python, caption={Multi-line Strings in Single or Double Quotes}, label={lst:multi-line}]
multi_line_single = 'Hello,\nWorld!'
multi_line_double = "Hello,\nWorld!"

print(multi_line_single)
print(multi_line_double)
\end{lstlisting}

Code \ref{lst:multi-line} shows examples of using multi-line strings in
single or double quotes in Python. In this example, the variables
\texttt{multi\_line\_single} and \texttt{multi\_line\_double} are assigned
string values using the escape character \texttt{\textbackslash n} to
represent a newline character.

\subsection{Comments}

Comments are used to explain the code and make it more readable. In Python,
comments start with the hash character (\texttt{\#}) and continue to the
end of the line. Comments are ignored by the Python interpreter and are
not executed as part of the program.

\begin{lstlisting}[language=Python, caption={Comments in Python}, label={lst:comments}]
# This is a single-line comment

'''
This is a multi-line comment.
It can span multiple lines.
'''

"""
This is also a multi-line comment.
It can span multiple lines.
"""
\end{lstlisting}

Code \ref{lst:comments} shows examples of single-line and multi-line
comments in Python. In this example, the single-line comment starts with
the hash character (\texttt{\#}), while the multi-line comment is enclosed
in triple quotes (\texttt{'''} or \texttt{"""}).

\subsection{Lines and Indentation}

Python uses indentation to define the structure of the code. Indentation
is used to group statements together. The number of spaces in the
indentation is not fixed, but all statements within the block must be
indented the same amount.

\begin{lstlisting}[language=Python, caption={Lines and Indentation in Python}, label={lst:indentation}]
if 5 > 2:
    print("Five is greater than two!")
\end{lstlisting}

Code \ref{lst:indentation} shows an example of using indentation in Python.
In this example, the \texttt{print()} statement is indented to indicate
that it is part of the \texttt{if} block. The number of spaces in the
indentation is not fixed, but all statements within the block must be
indented the same amount.

\subsection{Multi-Line Statements}

In Python, a statement can span multiple lines if it is enclosed in
parentheses \texttt{()}, square brackets \texttt{[]}, or curly braces
\texttt{\{\}}. This is useful when you have a long statement that you
want to split into multiple lines for readability.

\begin{lstlisting}[language=Python, caption={Multi-Line Statements in Python}, label={lst:multiline}]
numbers = [1, 2, 3, 4, 5,
           6, 7, 8, 9, 10]

total = (1 + 2 + 3 +
            4 + 5 + 6 +
            7 + 8 + 9 + 10)

colors = {'red': 255, 'green': 255,
            'blue': 255}
\end{lstlisting}

Code \ref{lst:multiline} shows examples of using multi-line statements in
Python. In this example, the list \texttt{numbers}, the sum \texttt{total},
and the dictionary \texttt{colors} are defined using multi-line statements
enclosed in square brackets, parentheses, and curly braces, respectively.

\subsection{Data Types}

In Python, every value has a data type. Data types are used to represent
different types of data, such as numbers, strings, lists, tuples, dictionaries,
etc. These data types are used to store, manipulate, and represent data in
Python programs. Say for example, the data type \texttt{int} is used to
represent integers, the data type \texttt{float} is used to represent
floating-point numbers, and the data type \texttt{str} is used to represent
strings. Using the correct data type is important because it determines
how the data is stored and how it can be manipulated.

\subsubsection{Integers}

Integers are whole numbers, such as 1, 2, 3, 4, 5, etc. In Python, integers
are represented using the \texttt{int} data type. Integers can be positive
or negative, and they can be used in mathematical operations such as addition,
subtraction, multiplication, and division.

\begin{lstlisting}[language=Python, caption={Integers in Python}, label={lst:integers}]
x = 5
y = -10

print(x)
print(y)

print(type(x))
print(type(y))
\end{lstlisting}

Code \ref{lst:integers} shows examples of using integers in Python. In this
example, the variables \texttt{x} and \texttt{y} are assigned integer values
5 and -10, respectively. The \texttt{print()} function is used to display
the values of the variables, and the \texttt{type()} function is used to
display the data type of the variables.

\subsubsection{Floats}

Floats are decimal numbers, such as 1.0, 2.5, 3.14, 4.0, etc. In Python,
floats are represented using the \texttt{float} data type. Floats can be
used to represent real numbers, and they can be used in mathematical
operations such as addition, subtraction, multiplication, and division.

\begin{lstlisting}[language=Python, caption={Floats in Python}, label={lst:floats}]
x = 3.14
y = -2.5

print(x)
print(y)

print(type(x))
print(type(y))
\end{lstlisting}

\subsubsection{Strings}

\subsubsection{Lists}

\subsubsection{Tuples}

\subsubsection{Dictionary}

\subsection{Conversion between Data Types}

\subsection{Basic Operators}

\subsubsection{Arithmetic Operators}

\subsubsection{Comparison Operators}

\subsubsection{Assignment Operators}

\subsubsection{Logical Operators}

\chapter{Control Statements}

\section{Introduction}

\section{Conditional Statements}

\subsection{If Statement}

\subsection{If-Else Statement}

\subsection{If-Elif-Else Statement}

\section{Looping Statements}

\subsection{For Loop}

\subsection{While Loop}

\subsection{Nested Loops}

\chapter{Functions}

\section{Introduction}

\section{Defining Functions}

\section{Calling Functions}

\section{Arguments and Parameters}

\section{Return Statement}

\section{Lambda Functions}

\chapter{File Handling}

\section{Introduction}

\section{Opening and Closing Files}

\section{Reading and Writing Files}

\section{Working with Directories}

\section{CSV File}

\subsection{Reading CSV Files}

\subsection{Writing CSV Files}

\subsection{CRUD Operations in CSV Files}

\subsubsection{Create Operation}

\subsubsection{Read Operation}

\subsubsection{Update Operation}

\subsubsection{Delete Operation}

\chapter{Exception Handling}

\section{Introduction}

\section{Try-Except Block}

\section{Finally Block}

\section{Raising Exceptions}

\section{User-Defined Exceptions}

\chapter{Object-Oriented Programming}

\section{Introduction}

\section{Classes and Objects}

\section{Inheritance}

\section{Polymorphism}

\section{Encapsulation}

\section{Abstraction}

\chapter{Modules and Packages}

\section{Introduction}

\section{Creating Modules}

\section{Importing Modules}

\section{Creating Packages}

\section{Importing Packages}

\chapter{Regular Expressions}

\section{Introduction}

\section{Match Function}

\section{Search Function}

\section{Findall Function}

\section{Split Function}

\section{Sub Function}

\chapter{References}

\begin{enumerate}[label={\textbf{\Alph*.}}]
  \item \textbf{Books}
    \begin{itemize}
      \item 
    \end{itemize}
  \item \textbf{Other Sources}
    \begin{itemize}
      \item 
    \end{itemize}
\end{enumerate}

\end{document}
